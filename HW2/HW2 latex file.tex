\documentclass{article}
\usepackage[utf8]{inputenc}

\title{HW2 Computer Org}
\author{Ayush Krishnappa}
\date{September 2022}

\begin{document}

\maketitle

\section{3.15.5}
\begin{itemize}
  \item a) Performance = $(\frac{Clock\ Rate}{CPI})$ instructions per sec\newline
  Processor P1 performance = $(\frac{3GHz}{1.5}) = (2 \times 10^9)$ instructions per sec\newline
  Processor P2 performance = $(\frac{2.5GHz}{1.0}) = (2.5 \times 10^9)$ instructions per sec\newline
  Processor P3 performance = $(\frac{4.0GHz}{2.2}) = (1.82 \times 10^9)$ instructions per sec\newline
  From these calculations we can see that P2 has the highest performance in terms of instructions per sec
  
  \item b) Number of cycles = $(clock\ rate \times 10)$\ cycles\newline
  Number of instructions = $\frac{(clock\ rate \times 10)}{CPI}$\newline
  Processor P1 cycles = $(3GHz \times 10) = (3 \times 10^{10})$ cycles\newline
  Processor P1 instructions = $\frac{(3GHz \times 10)}{1.5} = (2 \times 10^{10})$ instructions\newline
  Processor P2 cycles = $(2.5GHz \times 10) = (2.5 \times 10^{10})$ cycles\newline
  Processor P2 instructions = $\frac{(2.5GHz \times 10)}{1.0} = (2.5 \times 10^{10})$ instructions\newline
  Processor P3 cycles = $(4GHz \times 10) = (4 \times 10^{10})$ cycles\newline
  Processor P2 instructions = $\frac{(4.0GHz \times 10)}{2.2} = (1.82 \times 10^{10})$ instructions\newline
  
  \item c) Execution time = $\frac{Number\ of\ instructions \times CPI}{Clock\ Rate}$\newline
  Reduce exec time by 30 percent, increase of 20 percent in CPI value\newline
  $(Execution\ time \times 0.7) = \frac{Number\ of\ instructions\ \times \ CPI\ \times \ 1.2}{1.71 \times Clock\ Rate}$\newline
  New Clock Rate = $\frac{Clock\ Rate\ \times \ 1.2}{0.7}$\newline
  Processor p1 new clock rate = $(3GHz \times 1.71)$ = 5.13 GHz\newline
  Processor p2 new clock rate = $(2.5GHz \times 1.71)$ = 4.27 GHz\newline
  Processor p3 new clock rate = $(4GHz \times 1.71)$ = 6.84 GHz\newline
\end{itemize}

\section{3.15.7}
\begin{itemize}
  \item a) Formula for Global CPI = $\Sigma (CPI \times Fi)$, Fi = frequency count of each class\newline
  Global CPI for p1 = $(1 \times 0.1) + (2 \times 0.2) + (3 \times 0.5) + (3 \times 0.2) = (0.1 + 0.4 + 1.5 + 0.6) = 2.6CPI$\newline
  Global CPI for p2 = $(2 \times 0.1) + (2 \times 0.2) + (2 \times 0.5) + (2 \times 0.2) = (0.2 + 0.4 + 1 + 0.4) = 2CPI$\newline
  
  \item b) CPU clock cycles = $\Sigma (CPI \times Ci)$, Ci = instruction count\newline
  CPU clock cycles for P1 = $(1 \times 10^5) + (2 \times 2 \times 10^5) + (3 \times 5 \times 10^5) + (3 \times 2 \times 10^5) = 2.6 \times 10^6$\newline
  CPU clock cycles for P2 = $(2 \times 10^5) + (2 \times 2 \times 10^5) + (2 \times 5 \times 10^5) + (2 \times 2 \times 10^5) = 2 \times 10^6$\newline
  CPU execution time for P1 = $\frac{2.6 \times 10^5}{2.5 \times GHz} = 1.04 ms$\newline
  CPU execution time for p2 = $\frac{2 \times 10^5}{3 \times GHz} = 666.67 ms$\newline
  Therefore, processor p2 is faster
  
\end{itemize}

\section{3.15.8}
\begin{itemize}
  \item a) CPU time = $(instruction\ count \times CPI \times Clock\ cycle\ time)$\newline
  CPI = $\frac{cpu\ time}{(instruction\ count \times clock\ cycle\ time)}$\newline
  Compiler A CPI = $(\frac{1.1}{1.0E9 \times 1.0E-9)} = 1.1$\newline 
  Compiler B CPI = $(\frac{1.5}{1.2E9 \times 1.0E-9)} = 1.25$\newline
  
  \item b) Execution time = $Instruction \times CPI\ clock\ rate$\newline
  instructions1 x CP11 clock rate1 = instructions2 x CP12 clock rate1\newline
  $(\frac{10^9 \times 1.1}{1 \times 10^9 \times 1.25}) \times (clockrate2) = 0.73clockrate2$\newline
  clockrate1 = 0.73clockrate2\newline

  \item c) $CPU \times time_c = 0.66s$\newline
  $\frac{CPUtime_A}{CPUtime_C}$ = 1.67\newline
  $\frac{CPUtime_B}{CPUtime_C}$ = 2.27\newline
  Compiler C is 1.67 times faster then compiler A\newline
\end{itemize}

\section{3.15.10}
\begin{itemize}
    \item a) 1 processor execution time = $(\frac{(2.56E9 \times 1) + (1.28E9 \times 12) + (256 \times 1000000 \times 5)}{2 \times 1000000000}$ = 9.6 seconds\newline
    \ \\
    2 processor execution time = $(\frac{(2.56E9 \times 1) + (1.28E9 \times 12)}{0.7 \times 2}) + (256 \times 1000000 \times 5) -> (\frac{14080000000}{2 \times 1000000000})$ = 7.04 seconds\newline
    Relative speedup from 1 to 2 = $\frac{9.6}{7.04}$ = 1.36\newline
    \ \\
    4 processor execution time = $(\frac{(2.56E9 \times 1) + (1.28E9 \times 12)}{0.7 \times 4}) + (256 \times 1000000 \times 5) -> (\frac{7680000000}{2 \times 1000000000})$ = 3.84 seconds\newline
    Relative speedup from 1 to 4 = $\frac{9.6}{3.84}$ = 2.5\newline
    \ \\
    8 processor execution time = $(\frac{(2.56E9 \times 1) + (1.28E9 \times 12)}{0.7 \times 8}) + (256 \times 1000000 \times 5) -> (\frac{4480000000}{2 \times 1000000000})$ = 2.24 seconds\newline
    Relative speedup from 1 to 8 = $\frac{9.6}{2.24}$ = 4.3\newline
    
    \item b) 1 processor execution time = $(\frac{(2.56E9 \times 2) + (1.28E9 \times 12) + (256 \times 1000000 \times 5)}{2 \times 1000000000}$ = 10.88 seconds\newline
    \ \\
    2 processor execution time = $(\frac{(2.56E9 \times 2) + (1.28E9 \times 12)}{0.7 \times 2}) + (256 \times 1000000 \times 5) -> (\frac{15900000000}{2 \times 1000000000})$ = 7.95 seconds\newline
    \ \\
    4 processor execution time = $(\frac{(2.56E9 \times 2) + (1.28E9 \times 12)}{0.7 \times 4}) + (256 \times 1000000 \times 5) -> (\frac{8600000000}{2 \times 1000000000})$ = 4.3 seconds\newline
    \ \\
    8 processor execution time = $(\frac{(2.56E9 \times 2) + (1.28E9 \times 12)}{0.7 \times 8}) + (256 \times 1000000 \times 5) -> (\frac{4940000000}{2 \times 1000000000})$ = 2.47 seconds\newline
    \ \\
    
    \item c) The CPI of load/store instruction should be reduced by 25 percent\newline
    Using values from part a with 4 processors\newline
    3.84 = $\frac{(2.56E9 + 1.26E9 \times i) + (2.56E8 \times 5)}{2.0E9}$\newline
    \ \\
    i = $\frac{((3.84)(2.0E9) - (2.56E8 \times 5) - (2.56E9))}{1.26E9}$ = 3.047\newline
    \ \\
    3.047 / 12 = 0.25\newline
    
\end{itemize}

\section{3.15.12}
\begin{itemize}
    \item a) CPI = $\frac{(750 sec) \times (3 \times 10^9 s^-1)}{2.389 \times (10)^{12}}$ = 0.9427
    \item b) SPECratio = $\frac{reference\ time}{measured\ time}$ = $\frac{9650 sec}{750 sec}$ = 12.867
    \item c) New CPU time = (New instruction count) x (CPI) x (Clock cycle time) = (1.1)(old time) = $(1.1) \times (750)$ = 825 sec
    \item d) New CPU time = (1.1 x IC) x (1.05 x CPI) x (clock cycle time) = (1.55)(old time) = $(1.155) \times (750)$ = 866.25 sec, increase of 15.5 percent in CPU time
    \item e) change in SPECratio = (reference time) / (1.55 x measured time) = $\frac{12.867}{1.155}$ = 11.139
    \item f) CPI = $\frac{(700 sec) \times (4 \times 10^9 cycles*s^-1)}{(0.85)(2.389 \times (10)^{12})}$ = 1.38
    \item g) Change in CPI = $\frac{1.37 - 0.94}{0.94}$ = 0.43\newline
    Clock rate change = $\frac{(4.0 \times 10^9) - (3 \times 10^9)}{e \times 10^9}$ = 0.333\newline
    Instruction count has been reduced so the increase in CPI is similar to that of the clock rate.
    \item h) Percentage of CPU reduction time = $\frac{750 - 700}{750} \times 100 = 6.66\%$
    \item i) New execution time = $960 - (960 \times \frac{10}{100})$ = 864\newline
    Number of instructions = $\frac{864 \times 4 \times 10^9}{1.61} = 2.15 \times (10)^{11}$
    \item j) New number of instructions = $864 - (864 \times \frac{10}{100}) = 777.6$\newline
    New Clock rate = $\frac{2.146 \times 10^9 \times 1.61}{864}$ = 3.33 GHz
    \item k) Reduced Execution time = $960 - 0.2 \times 960$ = 768 sec\newline
    Reduced CPI = $1.61 - 0.15 \times 1.62$ = 1.37\newline
    Clock Rate = $\frac{2146 \times 10^9 \times 1.37}{768}$ = 3.82 GHz
\end{itemize}

\section{3.15.15} 
\begin{itemize}
    \item a) Clock Cycle = $(50 \times 10^6 \times 1) + (110 \times 10^6 \times 1) + (80 \times 10^6 \times 4) + (16 \times 10^6 \times 2) + (50 \times 10^6) + (110 \times 10^6) + (320 \times 10^6) + (32 \times 10^6) = 512 \times 10^6 s$\newline
    \ \\
    Execution time = clock cycle / clock rate = $\frac{512 \times 10^6}{2 \times 10^9} = 256 \times (10)^{-3}s$ 
    \ \\
    $CPI_FPimproved$ =  $\frac{\frac{512 \times 10^6}{2} - ((110 \times 10^6 \times 1) + (80 \times 10^6 \times 4) + (16 \times 10^6 \times 2))}{50 \times 10^6} = -4.12$\newline
    Since the value is negative, the CPI of FP instructions cannot be improved since the value might be negative when the program runs two times faster. 
    \item b) CPI (improved FP) =  $\frac{\frac{512 \times 10^6}{2} - ((50 \times 10^6 \times 1) + (110 \times 10^6 \times 1) + (16 \times 10^6 \times 2))}{80 \times 10^6} = \frac{512 - 944}{80} = 0.8$\newline
    For the program to run two times faster, must improve the CPI of L/S instruction as $\frac{4}{0.8} = 5$\newline
    Therefore, CPI of L/S instructions must improve by 5 times.
    \item c) Clock Cycles = $(50 \times 10^6 \times 0.6) + (110 \times 10^6 \times 0.6) + (80 \times 10^6 \times 2.8) + (16 \times 10^6 \times 1.4) = 342.4 \times 10^6$\newline
    Execution time = $\frac{342.4 \times 10^6}{2 \times 10^9} = 171.2 \times (10)^{-3}$\newline
    Improving execution time of program by $\frac{0.256}{0.171} = 1.497\ times$
\end{itemize}

\end{document}
